\documentclass[12pt]{article}
\usepackage[ngerman]{babel}
\usepackage[utf8]{inputenc}
\usepackage[T1]{fontenc}

\usepackage{amsmath,amssymb}

\newcommand{\pc}{\mathrm{pc}}

\begin{document}
	\setcounter{section}{5}
	\section{Bernards Stern}
	\subsection{Man berechne die Raumgeschwindigkeit des Sterns}
	Die Radialgeschwindigkeit relativ zur Sonne sei $v_0$, die Geschwindigkeit senkrecht dazu $v_1$.
	Für $v_1$ lässt sich als Geschwindigkeit $tan(\mu)*r$ abschätzen. Das ergibt $tan(\frac{10.3}{3600*3600*24*365})*1.82 \pc/s=1.65*10^{-10} \pc/s$.
	Die beiden Geschwindigkeitskomponenten werden gemäß dem Satz des Pythagoras addiert. $v_{ges}=\sqrt(v_0^2+v_1^2)$.
	\subsection{Man berechne den minimalen Abstand zur Sonne und die Zeit, die der Stern benötigt, um den minimalen Abstand zu erreichen}
	
	\subsection{Welche scheinbare Helligkeit wird der Bernard Stern im minimalen Abstand haben?}
	
	\section{Hübsches Duo}
	\subsection{Schätze mithilfe des Bildes den Radius der Venus ab}
	
	\subsection{Wann ist die nächste mögliche Mondfinsternis zu erwarten?}
	
	\section{Ein Doppelsternsystem}
	\subsection{Konstruiere den Schwerpunkt S des Systems und zeichne den Ort des schwarzen Lochs ein, die zur Position 2 des sichtbaren Sterns gehört}
	Geogebra zeichnung + screenshot davon
	
	\subsection{Schätze die Exzentrizität der Sternbahn ab}
	
	\subsection{Die Umlaufdauer des Systems ist $T = 80 d$. Berechne die Massen des Sterns und des schwarzen Lochs}
	
	\subsection{Berechne den kleinsten Abstand zwischen den Objekten des Systems (in Einheiten vom Schwarzschild-Radius des schwarzen Lochs)}
	
	\subsection{Schätze das Zeitintervall zwischen den Positionen 1 und 2 des Sterns (der Stern bewegt sich gegen den Uhrzeigersinn) ab}
	
	\subsection{Kann eine Akkretion im System stattfinden?}
	
	\section{Jeans-Kriterium und Kollabierende Wolke}
	\subsection{Leite die Gleichung für $E_{\mathrm{grav}}$ her}
	
	\subsection{Löse (1) nach $M_{\mathrm{Jeans}}$ einer molekularen Wasserstoffwolke (in Abhängigkeit von $\rho$ und $T$) auf}
	
	\subsection{Bestimme das effektive Potential $E_{\mathrm{pot}}^{\mathrm{eff}}$ für kollabierende Teilchen der Masse $m$ mit dem Drehimpuls $J$. Beim welchen Radius tritt ein Minimum auf?}
	
	\subsection{Genauere Modellbetrachtungen zeigen, dass bereits während des Kollapses ein Drehimpulstransport von innen nach außen geschieht. Wie kann das durch die Stöße der Teilchen erklärt werden?}
	
	\subsection{Ein weiterer wirksamer Mechanismus zum Transport vom Drehimpuls kann auch durch ein Magnetfeld geliefert werden. Sobald beim Kollaps durch die Temperaturerhöhung die neutralen Atome ionisiert werden, können die geladenen Teilchen das Magnetfeld des Sterns mitführen. Durch welche Kraft erfolgt dadurch eine Drehimpulserhöhung und warum weitet sich der Ring damit auf?}
\end{document}