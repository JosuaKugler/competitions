\subsection{Verse 847-853}
\begin{frame}
	\frametitle{Interpretation Vergil Buch 6, V.847-853}
	
	\alert<1-2,5>{excudent} \alert<3>{alii} spirantia mollius aera -\\
	credo equidem - vivos ducent de marmore voltus;\\
	\alert<1-2,5>{orabunt} causas melius caelique meatus\\
	\alert<1-2,5>{describent} radio et surgentia sidera dicent:\\\ \\
	\alert<3>{tu} regere imperio populos, Romane, memento -\\
	haec tibi \alert<2>{erunt} \alert<4>{artes} - pacique \alt<5>{\textcolor{blue}{inponere}}{inponere} morem\\
	\alt<5>{\textcolor{blue}{parcere}}{parcere} subiectis et \alt<5>{\textcolor{blue}{debellare}}{debellare} superbos.
	\ \\\ \\
	\only<1,2>{\invisible{Rückbezug und Gegenüberstellung\\}}
	\only<3>{Antithese}
	\only<4>{Rückbezug}
	\only<5>{Rückbezug und Gegenüberstellung}
\end{frame}