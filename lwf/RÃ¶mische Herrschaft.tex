% !TeX program = xelatex
\documentclass[12pt]{article}
\usepackage[a4paper,left=2cm,right=2cm,top=2cm,bottom=2cm]{geometry}
\usepackage[onehalfspacing]{setspace}
\usepackage[T1]{fontenc}
\usepackage[ngerman]{babel}
\usepackage[hidelinks]{hyperref}
\title{Römische Herrschaft}
\author{Josua Kugler}
\date{Alter Richener Weg 16\\ 75031 Eppingen}
\renewcommand{\familydefault}{\sfdefault}
\usepackage{fontspec}
%\setmainfont{Times New Roman}
\begin{document}
	\pagestyle{empty}
	\maketitle
	\thispagestyle{empty}
	\newpage
	\thispagestyle{empty}
	\tableofcontents
	\newpage
	\pagestyle{plain}
	\pagenumbering{arabic}
	\section{Einleitung}
	Die Vorstellungen der Antike, was Römische Herrschaft zu bedeuten hat, divergierten sehr stark. Auch hier finden sich beide Positionen gegenüber dem Großreich und der Ausdehnung Roms, die allen Völkern die \glqq pax romana\grqq\ bringen soll.
	Vergil widmete sich in der Aeneis, dem Nationalepos des römischen Volkes\footnote{Ebersbach 2007}, natürlich auch diesem Aspekt. Auch der von Tacitus biographisch porträtierte Agricola beschäftigt sich, genauso wie sein Gegner Calgacus, notwendigerweise mit diesem Thema, da der Zweck und Ausgang der ihnen bevorstehenden Schlacht eng mit dem Thema verbunden ist.
	\section{Vergil}
	Vergil führt in Vers 847 bis 850 des 6. Buches seiner \glqq Aeneis\grqq\ anhand von Kunst, Rhetorik und Wissenschaft drei Aspekte aus, die nicht primär zum Auftrag der Römer zählen. Zunächst einmal muss betont werden, dass, obwohl vermutlich die Griechen gemeint sind, kein bestimmtes Volk genannt wird. Stattdessen bleibt er mit dem Begriff \glqq alii\grqq(V. 847) sehr vage. Durch die Hyperbel \glqq spirantia\grqq(V. 847) charakterisiert Vergil die in der Hellenistik angestrebte Lebendigkeit der Skulptur. Dieser Aspekt wird auch im darauffolgenden Vers wiederholt und durch das alliterative Hyperbaton \glqq vivos [...] voltus\grqq(V. 848) betont. In der Tat bleibt die Ausstrahlung griechischer Skulpturen, abgesehen vielleicht von manchem Künstler der Renaissance, bis heute unerreicht. Jedoch wird die Superiorität der griechischen Bronzegießerei beziehungsweise Bildhauerei durch die Parenthese \glqq credo equidem\grqq(V. 848) relativiert. Die durch diese Unterbrechung eingeführte Subjektivität impliziert die Möglichkeit einer nur geringen Differenz zwischen der griechischen und römischen Kunst.
	Nichtsdestotrotz erhebt das römische Volk nicht den Anspruch, vollendeten Künstlern ebenbürtig zu sein. 
	
	In dieser Manier fährt Vergil in Bezug auf die Rhetorik fort. Durch das Adverb \glqq melius\grqq(V. 849) wird \glqq causas orare\grqq(ebd.) nicht ganz dem Kompetenzbereich des Römers entzogen. Dennoch wird auch hier auf die Inanspruchnahme einer führenden Position im Gebiet der Rhetorik verzichtet. Vers 849 ist durch eine Zäsur (Hepthemimeres) zwischen \glqq melius\grqq(V. 849) und \glqq caelique\grqq(V. 849) geprägt. Dadurch wird deutlich, dass Vergil sich mit einem dritten Aspekt beschäftigt, der durch das Enjambement am Versende(V. 849) in der nächsten Zeile fortgeführt wird. Die parallel angeordneten Verben \glqq excudent\grqq(V. 847), \glqq orabunt\grqq(V. 849) und \glqq describent\grqq(V. 850) spiegeln die drei Bereiche der Bildende Kunst, der Redekunst sowie der Kunst der Astronomie(die im Lateinischen als Wissenschaft und damit als eine der artes optimae gilt). Nicht am Versanfang, aber dennoch in der gleichen Form(3. Person Plural Futur 1 Indikativ Aktiv) stehen \glqq ducent\grqq(V. 848) und \glqq dicent\grqq(V. 848), wobei Ersteres noch der Bildenden Kunst, Letzteres bereits der Astronomie zugehörig ist. Wichtig für die Charakterisierung der Astronomie ist der Zeigestab \glqq radio\grqq(V. 850) als Symbol für die Präzision, mit der die Wissenschaft betrieben wird.
	
	Insgesamt wird von Vers 847 bis 850 deutlich, dass es primär nicht darum geht, wer die obigen, natürlich nur beispielhaften drei Künste besser beherrscht als die Römer, da jene unter dem Begriff \glqq alii\grqq(V. 847) zusammengefasst werden. Tatsächlich geht es darum, zu zeigen, dass die Römer nicht den Anspruch erheben oder erheben sollen, in einer derartigen Kunstform den vollendeten Künstlern ebenbürtig zu sein. Vielmehr sollen sie sich auf ihre \textit{artes} konzentrieren, die mit einem scharfen, vorangestellten \glqq tu\grqq(V. 851) in Kontrast zum vagen \glqq alii\grqq(V. 847) eingeleitet werden.
	
	An zweiter Stelle wird der Auftrag der Römer kurz zusammengefasst, er lautet: \glqq regere imperio populos\grqq(V. 851). Dass es sich hier um einen Auftrag handelt, wird sprachlich am feierlichen Imperativ \glqq memento\grqq(V. 851) sowie an der direkten Anrede mithilfe des Vokativ \glqq Romane\grqq(ebd.) deutlich.
	In der darauffolgenden Parenthese wird die Einordnung des römischen Herrschaftsauftrags in eine Reihe von Künsten explizit ausgesprochen: Zunächst schlägt das Verb \glqq erunt\grqq(V. 851) durch seine Form eine Parallele zu den Tätigkeitsverben der zuvorgenannten drei Künste. Daraufhin wird durch die Verwendung des Begriffs \glqq artes\grqq(V. 851) der römische Herrschaftsauftrag tatsächlich als Kunst aufgefasst.
	Durch das Demonstrativpronomen \glqq haec\grqq(V. 852)  werden die verschiedenen Aspekte dann von den anderen Künsten abgegrenzt und durch das \glqq tibi\grqq(ebd.) auf die Römer bezogen.
	
	Nun folgen drei weitere \textit{artes} der Römer, die durch das angehängte \glqq (paci)-que\grqq(V. 851) in ein Polysyndeton mit dem Herrschaftsauftrag gestellt werden. Daher sind sie nicht als Unterpunkte des \textit{regere}, sondern vielmehr als gleichwertige Künste aufzufassen. Dennoch sind sie natürlich inhaltlich abhängig von der Herrschaft der Römer. Interessanterweise kommen Eroberungen oder eine Erklärung, wie das \textit{Imperium Romanum} Macht über andere Völker erhält, überhaupt nicht vor. Stattdessen steht an zweiter Stelle eine Kunst, die sich ganz an Augustus orientiert. Mit \textit{pax} und \textit{mos} werden zwei große Schlagworte der augusteischen Politik zum zentralen Auftrag, ja sogar zur Kunst der Römer erhoben. Auch durch das sehr aktivisch konnotierte \glqq imponere\grqq(V. 851) wird die Philosophie der \textit{Pax Romana} bekräftigt. Es geht nicht darum, einen bereits vorhandene Frieden zu bewahren oder zu erhalten, sondern vielmehr darum, dass die Völker von Rom befriedet werden müssen.
	
	Die letzten beiden Aspekte sind formal sehr ähnlich aufgebaut. Zum einen zeichnen sie sich durch eine sehr parallele Satzstruktur aus, zum anderen beginnt das zu \glqq parcere\grqq(V. 853) gehörige Objekt mit \glqq sub\grqq(ebd.), das zu \glqq debellare\grqq(ebd.) gehörige Objekt mit \glqq sup(er)\grqq(ebd.): Zwei von der Aussprache her sehr ähnliche, von der Bedeutung aber antithetische Präfixe. Zudem tragen beide Aspekte gleichermaßen zum Frieden und zur Gerechtigkeit bei. Das Schonen der Unterworfenen beugt einem Aufstand vor, genauso wie das Niederkämpfen der \glqq superbos\grqq(V. 848) vor Überheblichkeit und damit vor Unfrieden warnt. Ein nicht zu vernachlässigender Aspekt besteht außerdem darin, dass \glqq debellare\grqq(V. 853) als erstes mit Gewalt verbundenes Wort erst an vierter Stelle auftaucht. Damit erhält die Gewalt eine starke Konnotation von Unwesentlichkeit, wodurch es gelingt, Gewalt als friedensbringendes beziehungsweise erhaltendes Instrument zu charakterisieren.
	\section{Tacitus}
	\subsection{Die Rede des Calgacus}
	\label{calgacus}
	Tacitus gibt in den Kapiteln 30-32 seines \glqq Agricola\grqq die Rede eines britannischen Heerführers vor der Schlacht gegen Agricola wieder. Calgacus, so der Name dieses Anführers, beginnt seine Rede mit einer sehr optimistischen Aussage. Dabei lobt er die Einmütigkeit der Britannier und beschreibt einen sehr wahrscheinlichen Sieg als Folge dieser Einheit. Den Sieg in der Schlacht sieht er als den Auslöser für die Freiheit Britanniens. Durch die Verwendung des Wortes \glqq libertas\grqq\ ganz zu Beginn der Rede erhebt er es zu einem seiner zentralen Schlagworte. Er stellt nämlich die \textit{servitutem} unter den Römern der \textit{libertas} ohne sie entgegen. Nachdem er auf diese Art und Weise seinen Soldaten Mut gemacht hat, erklärt er die Alternativlosigkeit des Krieges. Dadurch erreicht er, dass die Soldaten theoretisch keinen Grund dazu haben, nicht bis zum Tod zu kämpfen. Wer sowieso stirbt, falls er die Schlacht verliert, der kann kämpfen, da er dann entweder die oben beschriebene Freiheit erlangt oder zumindest eines ehrenvollen Todes stirbt.
	Im nächsten Abschnitt(30,2) geht Calgacus auf die Vergangenheit ein. Durch das Attribut \glqq varia\grqq\ zu \glqq fortuna\grqq\ zeigt er auf, dass die Römer nicht unbesiegbar sind, dass bereits andere  zwar teils von den Römern besiegt worden sind, teils aber auch die Römer besiegt haben. Da seine Soldaten aber die Edelsten (\glqq nobilissimi totius britanniae\grqq) seien, hätten sie folglich eine hohe Gewinnchance.
	
	Bereits hier lässt sich ein klares, psychologisch sehr durchdachtes Schema für seine Rede erkennen. Calgacus wechselt Abschnitte, die durch ihren Optimismus und ihre Siegesgewissheit Kriegslust wecken sollen mit Abschnitten, die durch Furcht vor der schrecklich dargestellten Niederlage zum Kämpfen bis zum Tod animieren sollen, da die Niederlage sowieso zum Tod oder zumindest zu einem lebensunwerten Leben führen würde.
	
	Nun fährt er, gemäß des Schemas, mit Begriffen wie \glqq servientium litora\grqq\ oder \glqq dominationis inviolatos\grqq(30,2) fort, die in starker Antithese zur Freiheit stehen. Zusätzlich zur \textit{libertas}-\textit{servitutis}-Antithese greift Calgacus in diesem Abschnitt die räumliche Ausweglosigkeit auf, die auch metaphorisch für die generelle Ausweglosigkeit ihrer Situation steht. Um seinen Soldaten wirklich Angst vor einer Niederlage zu machen, muss er die Römer als schreckliches Volk darstellen. Diese Absicht macht er mit dem Komparativ \glqq infestiores\grqq(30,3) klar; noch schlimmer als die Ausweglosigkeit sind also die Römer, die im Folgenden charakterisiert werden. Dass es einem nichts nützt, \textit{obsequium} oder \textit{modestia}(30,3) zu üben, impliziert eine starke Ungerechtigkeit der Römer, die durch das ihnen zugeordnete Objekt \glqq superbiam\grqq\ noch verstärkt wird. Neben diesen Eigenschaften ist auch noch der \glqq affectus\grqq, hier vor allem im Sinne von Gier sehr wichtig. Diese Charakterisierung wir vor allem durch bestimmte Begriffe sehr deutlich, wie z.B. \glqq raptores orbi\grqq, \glqq avari\grqq\  als Gier nach Geld, \glqq ambitiosi\grqq(30,4) als Gier nach Ruhm, ihre sogar durch eine Antithese hervorgehobene Unersättlichkeit (\glqq non oriens, non occidens satiaverit\grqq(30,4)) genauso wie schließlich \glqq pari adfectu\grqq(30,4) als adverbiale Bestimmung beschreibt, wie sich die Römer auf sowohl Armut als auch Reichtum stürzen, was durch die Antithese und das Prädikativum \glqq soli omnium\grqq(30,4) hyperbolisch wirkt.
	
	Tacitus' Calgacus kannte sich in der römische Politik oder sogar Ideologie aus, da er diese nun in einem Satz niedermacht. Nach einem asyndetischen Trikolon dreier äußerst negativ konnotierter Infinitive (\glqq auferre\grqq, \glqq trucidare\grqq, \glqq rapere\grqq) fasst er diese als das römische Verständnis von Herrschaft (\glqq imperium\grqq(30,4)) zusammen und beschreibt die Umsetzung der \textit{pax romana} als Einöde.
	
	Danach (31,5) geht er auf die Ausbeutung der von den Römern Unterworfenen ein. Die Römer würden die Familien zerreißen, um diese als Sklaven einzusetzen. Laut Calgacus schänden sie die Frauen, nicht nur in Kriegs- sondern auch in Friedenszeiten, was in starker Antithese zu jeglicher Form von \textit{mos} steht. Schließlich werde der gesamte Besitz, sogar der eigene Körper als Gipfelpunkt dieser Reihung von den Römern beansprucht. Nun greift Calgacus, um die sittenlose, gierige Ausbeutung der Römer angemessen beschreiben zu können, zu einem Gleichnis. Anstelle des bei Sklavenbesitzern üblichen Verfahrens, gemäß welchem die Römer ihre Sklaven einmal kaufen und dann selbst ernähren würden, muss in diesem Fall Britannien seine eigene Knechtschaft, natürlich hyperbolisch, täglich (\glqq cottidiae\grqq(31,2)) kaufen und zusätzlich selbst ernähren! Die Römer als Sklavenherren der ganzen Welt (vgl. \glqq raptores orbis\grqq) charakterisierend, vergleicht er Britannien mit einem neu hinzugekommenen Sklaven, der natürlich von allen anderen Sklavenvölkern misshandelt werde. Da die Britannier nicht reich sind, werde Rom seine Gier bereits am Ruhm gesättigt haben, sie besiegt zu haben. Danach gebe es keinen Grund mehr, sie am Leben zu lassen (\glqq reservemur\grqq).
	der Sinn dieses Abschnitts besteht darin, zu zeigen, dass im Falle der Niederlage die vollständige Vernichtung steht und es daher, wie oben erläutert, keinen Sinn macht, nicht bis zum Tod zu kämpfen. Bevor er selber dieses Fazit zieht, erörtert er noch einen weiteren Aspekt, der zur vollständigen Vernichtung des Volks führen werde: Die \textit{virtus} sowie \textit{ferocia}(31,3) der Unterworfenen gefalle den Römern nicht. Neben dem zusätzlichen Seitenhieb auf die Römer, die offensichtlich Tugend als lästig empfinden, flicht er hier sehr elegant und rhetorisch geschickt ein Lob auf seine Soldaten ein, das auch den Nationalstolz wecken soll.
	Schließlich fasst er den gesamten Abschnitt zusammen, indem er erklärt, was er damit bezwecken wollte: Bei den Römern gibt es keine Hoffnung auf Gnade. Daraus zieht er die Schlussfolgerung, dass es keine Alternative zum Kampf bis in den Tod gebe.
	
	Da Angst zwar ein sehr starkes psychologisches Fundament für ein Krieg darstellt, die Kriegslust aber eher durch Optimismus und Siegesgewissheit verstärkt wird, wendet er sich nun (31,4) der Frage zu, warum sie die Römer besiegen werden.
	Zunächst argumentiert er, dass die Briganten (aus damaliger Sicht \textit{sogar}) unter Führung einer Frau gegen die Römer Erfolge erzielen konnten. Durch die parallele Klimax \glqq exurere\grqq, \glqq expugnare\grqq, \glqq exuere\grqq\ wird dieser Vorgang charakterisiert, allerdings steht \textit{exuere} im Konjunktiv, da der Boudicca-Aufstand fehlgeschlagen ist. Dies erscheint aber zunächst nebensächlich, da die Reihung so parallel und sogar alliterativ verläuft. Dadurch werden die Römer natürlich entehrt, da sie ohne die Unvorsichtigkeit ihrer Gegner von einer Frau besiegt worden wären. Zum anderen schafft es den Soldaten Siegesgewissheit, da sie kaledonische Männer sind, die in Freiheit und Unberührtheit von römischer Knechtschaft leben, was in Antithese zum Boudicca-Aufstand steht.
	Die gerade angesprochene Ehr- und Sittenlosigkeit der Römer wird nun (32,1) weiter verstärkt. Im Frieden seien die Römer zügellos, wobei \glqq lascivium\grqq\ das exakte Gegenteil zu \textit{mos} darstellt. Im Kampf dagegen mangelte es ihnen an \textit{virtus}, was eine sehr schwerwiegende Beleidigung für jeden stolzen Soldaten darstellt.
	Die vermeintliche Ehre des römischen Heeres sei dementsprechend nur durch die Fehler ihrer Feinde zustande gekommen. Dieses Heer, der Stolz des \textit{Imperium Romanum}, sei außerdem nur ein zusammengewürfelter Haufen, dessen Zusammenhalt anstatt durch tugendhafte Treue zu Rom allein durch Angst geschaffen werde(32,2), dessen Kampfesmotivation durch die mangelnde Bindung an ein Heimatland und durch die Unkenntnis der Gegend in ein Zittern (\glqq intrepidus\grqq) verwandelt werde. Die Götter hätten dieses Heer gleichsam geopfert (\glqq clausos \dots ac vinctos\grqq) und den Britannieren ausgeliefert (32,2).
 	Insgesamt werde das römische Heer durch das Symbol von \glqq aurum\grqq\ und \glqq argentum\grqq(32,3) sehr gut zusammengefasst: Der Glanz sei nur eine untugendhafte Täuschung, die \glqq neque tergit neque vulnerat\grqq, d.h., durch die Antithese hervorgehoben, weder zur Defensive noch zur Offensive tauge. Bald würden die versklavten Völker ihre frühere \textit{libertas} suchen und also zu den freien Britanniern überlaufen.(32,3)
	
	Das \textit{Imperium Romanum} sei nur eine altersschwache Institution (vgl. senum, (32,3)), die von schlechtem Gehorsam und Ungerechtigkeit geprägt sei: Beide Aspekte stehen konträr zu sowohl \textit{mos} als auch \textit{virtus}, den Idealen des \textit{Imperium Romanum}.
	
	Nachdem Calgacus die Gegner also verbal vollständig auseinandergenommen hat, fasst er zusammen(32,4). Durch Verwendung der Demonstrativpronomen (bzw. bei ibi eigentlich Demonstrativpronominaladverbien) \glqq hic\grqq\ und \glqq ibi\grqq\ schafft er eine starke Polarisierung, wobei auf der einen Seite \glqq dux\grqq\ und \glqq exercitus\grqq stehen, auf der anderen Seite ewige Knechtschaft. Nach demselben Schema fortfahrend, eröffnet er die Wahl zwischen dem passivisch konnotierten Ertragen der Strafen oder dem aktivisch konnotierten \glqq ulcisci\grqq. Indem er die Soldaten auffordert, sowohl der Vorfahren als auch der Nachkommen zu gedenken, schafft er, genauso wie durch die Präzisierung \glqq hoc campo\grqq, der Schlacht einen temporal sowie lokal sehr exponierten Standpunkt.
	
	In der Beschreibung der Reaktion der Britannier (33,1) werden diese als Barbaren stereotypisiert und durch Geräusche wie Lärmen (\glqq fremitu\grqq), oder sogar, sehr negativ ausgedrückt, \glqq clamor dissonus\grqq, charakterisiert. Dagegen wird bereits im Vorwort (33,1) zu Agricolas Rede das römische Heer sehr tugendhaft dargestellt: Seine Soldaten sind bereits äußert kampflustig, bevor Agricola seine Rede überhaupt erst beginnt.
	\subsection{Die Rede des Agricola}
	\label{agricola}
	Agricola beginnt seine Rede, wie Calgacus, sehr optimistisch. Durch die Perfektform \glqq vicistis\grqq\ will er zeigen, dass die Eroberung Großbritanniens eigentlich schon vollendet ist.(33,2) Zudem stellt er sich durch den Begriff \glqq commilitones\grqq\ auf die gleiche Stufe mit seinen Soldaten und lobt sowohl ihre Tugend als auch die günstigen Rahmenbedingungen des \textit{Imperium Romanum}. Um die Gemeinschaft noch hervorzuheben, verwendet er das Wort \glqq nostra\grqq in Bezug auf zwei weitere sehr positive Eigenschaften seiner Soldaten: \textit{fide} und \textit{opera}. Diese Eigenschaften führt er weiter aus und dramatisiert \textit{opera} durch die Hyperbel, dass sie gegen die Natur selbst kämpfen hätten müssen. \textit{Fides} wird dadurch weiter bestärkt, dass weder Feldherr noch Heer mit dem jeweils anderen unzufrieden gewesen wäre. Als Folge dieser gemeinsamen Leistungen konkludiert er einen bisher unerreichten Erfolg. Dabei ist er von einer solchen Siegesgewissheit durchzogen, dass er von der Unterwerfung Britanniens in der Perfektform (\glqq subacta\grqq(33,3)) spricht. Diese Implikation, dass aus \textit{virtus} Erfolg resultiert, entspricht der Grundphilosphie des römische Reiches.
	Interessant ist in Agricolas Rede, dass er, genau wie Calgacus, alles auf diese Schlacht zuspitzt. Das auf dem beschwerlichen Marsch oft zu hörende \glqq Quando\grqq\ der Soldaten erfährt seine Antwort in der Präsensform \glqq veniunt\grqq(33,4). Die Bedeutung dieser Schlacht wird, wie bei Calgacus hyperbolisch ins Unendliche gesteigert:
	Durch das \glqq omnia\grqq\ bekommt der Satz zunächst eine starke Gewichtung. Agricola macht mithilfe des Chiasmus \glqq prona victoribus [atque eadem] victis adversa\grqq, wobei \textit{victoribus} und \textit{victis} aus dem selben Wortstamm kommen, ersteres aber aktivisch, letzteres passivisch konnotiert ist, stark polarisierend (vgl. Calgacus) die zwei Möglichkeiten des Schlachtausgangs deutlich. Dies soll den Soldaten klar machen, dass es sich eher lohnt, für einen Sieg ehrenvoll zu sterben, als zu fliehen(33,5). Um zu erklären, warum sich alles gegen die Besiegten wendet, zählt er die Nachteile auf, die den Römern durch ihre Situation im Falle einer Flucht entstehen, bevor er den Soldaten noch einmal explizit erklärt, was implizit aus dem Gesagten hervorgeht. Damit bedient er sich desselben rhetorischen Mittels wie Calgacus. Er führt etwas aus, bei dem der Hörer eine bestimmte Schlussfolgerung für sich zieht. Dann spricht er genau diese von ihm intentional eingeflochtene Schlussfolgerung aus, sodass jedem notwendigerweise die zunächst selbst gefundene und damit für einen eigenen Gedanken gehaltene Schlussfolgerung eingeprägt wird. Als nächstes greift Agricola wieder die seine gesamte Rede durchdringende Idee auf, alles auf die Schlacht zuzuspitzen, indem er diesen Ort als \glqq terrarum ac naturae fine\grqq(33,6) beschreibt. Damit gewinnt die Schlacht eine dramatische Dimension. Allerdings misst er mit dieser Beschreibung nicht nur der Schlacht eine zentrale Bedeutung, sondern auch dem \textit{Imperium Romanum} den Anspruch, die \textit{fines imperii} bis an die Enden der Erde auszudehnen, zu. (vgl. Vergil: \textit{imperium sine fine})
	Ab hier greift Agricola nicht mehr zu der Strategie, die Soldaten durch Angst vor der Niederlage zum Kämpfen bis zum Tod zu bringen, sondern versucht seine Soldaten durch Lob und das Verhöhnen der Britannier auf einen Sieg auszurichten und ihre Siegesgewissheit beziehungsweise ihr Selbstvertrauen zu stärken. Zunächst erinnert er sie an ihre \glqq decora\grqq, dann geht er genauer auf die Britannier ein: Diese seien durch Geschrei zu besiegen und also sehr schreckhaft. Durch den Vergleich mit einem Dieb (\glqq fur\grqq) wird ihnen ihre Ehre genommen. Anstatt die Tatsache, dass diese Britannier noch leben, durch ihre Tapferkeit zu erklären, wird sie als Folge dessen, dass sie am besten fliehen konnten, gedeutet, was natürlich große Schande über die Britannier bringt. 
	Agricola rechtfertigt dies(34,2) mit einem Vergleich, in welchem er die Britannier mit Tieren vergleicht, was per se schon herabwürdigend ist: Die mutigsten Tiere stürzten einem zuerst entgegen, während die ängstlichen und schüchternen Tiere natürlich fliehen. Übertragen auf die Britannier, sind die Übriggebliebenen also \glqq ignavi\grqq\ und \glqq timenti\grqq.
	Agricola fasst dies alles zusammen(34,3) und stellt antithetisch das erbärmliche Heer der Britannier dem großartigen Sieg der Römer gegenüber. Abschließend greift er wieder den Aspekt der exponierten Stellung dieser Schlacht auf. Durch die leicht parallel aufgebauten Sätze, die jeweils mit einem Imperativ im Plural beginnen, baut er Spannung auf. Zunächst einmal sollen die Soldaten ihre eigenen Feldzüge abschließen, dann sollen sie die fünfzig Jahre währende Eroberung Britanniens mit einem einzigen Tag abschließen; dies entspricht der Zuspitzung der Geschichte auf diese Schlacht. Als Höhepunkt der Klimax wird die \textit{res publica} mit einbezogen. Diese Schlacht soll in Verantwortung gegenüber dem Heer und dem Staat geführt werden, wodurch sie staatspolitische Dimension gewinnt und also dem Zweck des \textit{Imperium Romanum} dienen soll.
	\subsection{Vergleich der Rede des Calgacus und des Agricola}
	\label{calgacus-agricola}
	Beide Heerführer wollen die Soldaten zum einen damit motivieren, bis zum Tod zu kämpfen, indem sie ihnen die Situation nach einer Niederlage so schrecklich darstellen, dass es keinen Sinn macht zu fliehen, da man in der Schlacht zumindest eines ehrenvollen Todes sterbe. Zum anderen versuchen beide die Kampfeslust der Soldaten zu stärken, indem sie ihnen Siegesgewissheit verschaffen. Diese Gewissheit wird durch Lob der Soldaten, Erinnerung an deren Erfolge oder andere motivierende Ereignisse sowie möglichst schlechte Darstellung des Gegners erreicht. Bei Calgacus ist Ersteres stärker hervorgehoben, bei Agricola Letzteres. Das gibt ein Stück weit Aufschluss über die Erwartungen der Feldherren bezüglich des Schlachtausgangs: Agricola zeigt sich wesentlich optimistischer, was vermutlich in den vielen bereits selbst mit seinem Heer gewonnenen Schlachten seine Ursache hat, wie er selber argumentiert. Auch Calgacus bezieht die Vergangenheit mit ein, indem er den Boudicca-Aufstand erwähnt, da er mit seinem Heer noch keine Erfolge gegen die Römer verzeichnen kann. Interessanterweise erwähnt Agricola genau dies, nämlich dass er sein Heer \glqq aliorum exercituum exemplis\grqq\ ermutigen würde, wenn sie keine eigenen Erfolge erzielt hätten. Zusätzlich zu diesen eigenen Erfolgen kann Agricola aber auch auf eine lange Tradition römischer Feldherren im Verlauf des \textit{Imperium Romanum} zurückschauen und betont diesen Aspekt der Einbindung in eine größere Sache mehrmals.
	Sowohl die Rede des Calgacus als auch die des Agricola ist nämlich von dem Ziel durchzogen, der Schlacht möglichst große Bedeutung beizumessen, um den Kampfeswillen der Soldaten zu steigern. 
	Dies geschieht zum einen durch eine lokale und temporale Zuspitzung auf diese Schlacht, sodass ihr eine exponierte Stellung zukommt. Beispielsweise betonen beide Redner, dass die Schlacht am (geografischen) Ende der Welt stattfindet. Bei Calgacus steht das im Kontext der Glorifizierung seiner Soldaten, die als das letzte freie Volk am Ende der Welt das letzte Ziel der römischen Räuber seien. Agricola sieht sein Heer und sich selber als die mutigsten Eroberer aller Zeiten, die nun die letzten und entlegensten Gebiete der Erde erobern und damit weitergegangen sind als alle Heere und Heerführer vor ihnen. 
	Zum anderen erzielen sie eine Hervorhebung der Wichtigkeit dieser Schlacht, indem sie  diese einer übergeordneten Sache unterstellen. Diese übergeordnete Sache, in der die Soldaten den Sinn der Schlacht finden sollen, ist bei Calgacus die Freiheit (\textit{libertas}) vom \textit{Imperium Romanum}, bei Agricola geht es darum, den Herrschaftsauftrag des \textit{Imperium Romanum} umzusetzen. 
	Dementsprechend findet sich bei Calgacus eine teilweise sogar hyperbolisch negative Darstellung der römischen Herrschaft, in der er die Ideale und Grundgedanken des \textit{Imperium Romanum} untergräbt und kritisiert, während sich bei Agricola als Repräsentant des römischen Imperialismus eine positive Darstellung der römischen Herrschaft findet.
		
	\section{Vergleich des Ausschnitts aus Vergils \glqq Aeneis\grqq\ und des Ausschnitts aus Tacitus' \glqq Agricola\grqq}
	\label{vergil-tacitus}
	Bei dem Ausschnitt aus der \glqq Aeneis\grqq\ handelt es sich um eine sehr philosophische Definition des \textit{Imperium Romanum}. Die der augusteischen Politik entsprechende Sentenz \glqq pacique inponere morem\grqq\ beschreibt gleich zwei Ideale: die \textit{pax Romana}, ein Frieden, der die ganze Welt umspannen soll, dazu kommt \textit{mos} als Überbegriff für die römische Sitte. Dazu zählen natürlich ehrenhaftes Verhalten im Alltag, aber auch in der Politik und im Heer, wobei dieser Aspekt sich vor allem in der Disziplin manifestiert. 
	Calgacus, der seine Soldaten zum Kämpfen motivieren möchte, hat in doppelter Hinsicht Grund, die Römer schlecht darzustellen. Dabei zielt er zum einen darauf ab, Schlechtigkeit im Sinne von Bosheit auf die Römer zu projizieren, was die oben erläuterten psychischen Auswirkungen auf die Soldaten haben soll. Zum anderen möchte er die Siegesgewissheit seiner Soldaten stärken, indem er die Römer als schlechte Kämpfer darstellt, was aber für unsere Betrachtungen eher von geringer Relevanz ist.
	Den ersten Aspekt greift er vor allem im ersten Teil seiner Rede(30,31) auf. Die gesamten negativen Charaktereigenschaften, die unter \ref{calgacus} aufgeführt sind, stellen die Römer als ein Volk dar, dass vor lauter Gier von Sittenlosigkeit geprägt ist, was in Antithese zum Ideal \glqq mos\grqq\ steht. Zusätzlich unterdrücken laut Calgacus die Römer ihre Unterworfenen wie Sklaven, was in starker Antithese zu \glqq parcere subiectis\grqq\ steht. Dass die Römer Überheblichkeit (\glqq superbia\grqq) bestrafen wird dadurch ins Lächerliche gezogen, dass Calgacus unter Verwendung exakt derselben Vokabel (\glqq superbia\grqq) die Römer selbst als überheblich beschreibt. 
	Agricola hingegen hat sich zum Ziel gesetzt, den Herrschaftsauftrag des \textit{Imperium Romanum} umzusetzen. Daher unterstützt er natürlich dessen Idee und stellt es positiv dar, wobei in seiner Rede vor allem der kriegerische Aspekt betont wird. In der idealisierten Version Vergils wird Gewalt erst im letzten Punkt, dem Bezwingen der Überheblichen, angedeutet. Obwohl er damit bereits hier als Mittel zur Friedensschaffung bzw. -sicherung gedeutet wird, erscheint er doch eher nebensächlich. Zu Beginn seiner Rede hebt Agricola das \textit{Imperium Romanum} vor allem als günstige Rahmenbedingung für einen erfolgreichen Krieg hervor. Später greift er durch das Motiv des Endes der Welt den Gedanken eines \glqq imperium sine fine\grqq\ auf. Am Ende seiner Rede bezieht er die \textit{res publica} mit ein, die er als dem Heer übergeordnete Instanz, der es zu gefallen gilt, darstellt. Insgesamt geht aus seiner Rede eine positive Einstellung zum \textit{Imperium Romanum} hervor, wobei er vor allem Tapferkeit und Ehre als erstrebenswert sieht. Ihm geht es weniger um das Schonen von Unterlegenen oder um den Frieden, sodass er vermutlich eine realistischere Vorstellung als Vergil hat. Wie sollen die Römer denn ein \glqq imperium sine fine\grqq\ errichten, ohne den Krieg als Mittel zum Zweck zu verwenden?
	\section{Schluss}
	Insgesamt lässt sich konkludieren, dass es sehr unterschiedliche Auffassungen von Römischer Herrschaft gibt. Die idealistische, philosophische Grundidee des Vergil in seinem Gründungsepos des römischen Volkes beschreibt den Auftrag der Römer damit, Frieden mit Sitte einzusetzen und dabei die Schwachen zu stärken sowie die Überheblichen zu strafen. Agricola als Vertreter des römischen Heers und damit einer der Hauptakteure des \textit{Imperium Romanum} vertritt dahingegen eine deutlich realistischere Idee des römischen Reiches ein. Er räumt dem Krieg als notwendiges Mittel des \textit{Imperium sine fine} einen wesentlich höheren Stellenwert ein als in der idealisierten Fassung.  
	Von seinen Gegnern wird das \textit{Imperium Romanum} nicht als positiv empfunden: Völker, denen der römische Friede erst noch gebracht werden soll, fassen die Herrschaft der Römer als Sklavenherrschaft auf, die alle ihre Ideale verfehlt und eine pervertierte Form von Frieden auf der ganzen Welt implementiert.
	Insgesamt stellen sich zwei Fragen:\\
	Heiligt der Zweck des Friedensschaffens das Mittel eines Krieges?\\
	Kann ein Krieg überhaupt zu Frieden führen beziehungsweise lässt sich ein Frieden von oben durch Gewalt einsetzen (\glqq inponere\grqq)?
	\section{Gegenwartsbezug}
	In der Geschichte kam es oft vor, dass Völker eine sehr ideale Vorstellung von ihrer Herrschaft hatten. Beispielsweise war ab Mitte des 20. Jahrhunderts in Amerika die Vorstellung weit verbreitet, dass es die Aufgabe der Amerikaner sei, allen anderen Frieden und Fortschritt zu bringen und sich auf die ganze Welt auszubreiten. Spiegelt sich diese als \glqq Manifest Destiny\grqq\ bekannte Idealisierung nicht selbst heute noch teilweise wieder?\footnote{Kinzer 2006}\\
	Vor allem in Bezug auf militärische Operationen im Nahen Osten wird diese Vorstellung sehr häufig kritisiert, nicht nur bei den USA, sondern auch bei Russland oder sogar bei Einsätzen der deutschen Bundeswehr. Hier wurde oftmals der Krieg als Mittel zur Friedensschaffung propagiert, um militärische Interventionen zu rechtfertigen.
	Dabei wurde die Gesamtsituation im Nahen Osten meistens deutlich verschlechtert, anstelle des versprochenen Friedens traten Terror und eine vom Krieg schwer gezeichnete Region.
	
\newpage
\thispagestyle{empty}	
\section*{Literatur}
\subsection*{Primärliteratur}
P. Cornelius Tacitus: Agricola. Lateinisch/Deutsch. Übersetzt, erläutert und mit einem
Nachwort herausgegeben von Robert Feger, Ditzingen 1973

\noindent
Vergil: Aeneis. Lateinisch- Deutsch. In Zusammenarbeit mit Maria Götte hrsg. u.
übers. von Johannes Götte. München 1955
\subsection*{Sekundärliteratur}
Kinzer, Stephen: Overthrow: America‘s Century of Regime Change from Hawaii to
Iraq. New York 2006 (zitiert als: Kinzer 2006)

\noindent
P. Vergilius Maro: Aeneis. Prosaübertragung. Nachwort und Namensverzeichnis von
Volker Ebersbach. Ditzingen 2011, 393-427 (zitiert als: Ebersbach 2007)
\end{document}