\subsection{Verse 851-853}
\begin{frame}
	\frametitle{Interpretation Vergil Buch 6, V.851-853}
	\only<1-5,7->{\alert<3>{tu} \alt<4>{\textcolor{red}{regere}}{\alt<1>{\textcolor{blue}{regere}}{regere}} \alert<5>{imperio} \alert<4>{populos}, \alert<3>{Romane}, \alert<2>{memento}}
	\only<6>{\textcolor{blue}{tu} \textcolor{green}{regere} \textcolor{red}{imperio} \textcolor{green}{populos}, \textcolor{blue}{Romane}, memento}
	\ \\\ \\
	-haec tibi erunt artes- pacique \alt<1>{\textcolor{blue}{inponere}}{inponere} morem,\\
	 \only<1>{\textcolor{blue}{parcere} subiectis et \textcolor{blue}{debellare} superbos}
	 \only<2->{\alert<7>{parcere \alert<8>{sub}iectis} et \alt<7>{\textcolor{blue}{debellare superbos}}{debellare \alt<8>{\textcolor{blue}{sup}}{sup}erbos}}
	\\\ \\\ \\
	\only<1-5>{\invisible{Antithese}}
	\only<6>{abbildende Wortstellung}
	\only<7>{Antithese}
	\only<8>{Antithese, verstärkt durch Paronomasie}
\end{frame}