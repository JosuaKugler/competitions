\documentclass{article}
\usepackage[utf8]{inputenc}
\usepackage[T1]{fontenc}
\usepackage[ngerman]{babel}

\usepackage{amsmath,amssymb,amstext,amsthm}
\usepackage[fleqn,tbtags]{mathtools}
\numberwithin{equation}{section}

\usepackage{fancyhdr}
\pagestyle{fancy}
\rhead{Brunner, Kugler}
\usepackage{subcaption}
\usepackage{tikz,pgf,pgfplots}
\usetikzlibrary{angles,quotes,babel,arrows,intersections,positioning}
\pgfplotsset{compat=1.15}
\usepackage{mathrsfs}
\newcommand{\degre}{\ensuremath{^\circ}}
\newtheorem{df}{Definition}
\newtheorem{satz}{Satz}
\newtheorem{lemma}{Lemma}
\newtheorem{fakt}{Fakt}
\newtheorem{korollar}{Korollar}
\newcommand{\RM}[1]{\MakeUppercase{\romannumeral #1{.}}}
%\newcommand{\winkel}{30}

\title{Bundeswettbewerb Mathematik 2019 Runde 1}
\author{Julian Brunner und Josua Kugler}

\begin{document}
	\maketitle
	\newpage
	\section{Aufgabe}
	Wir versuchen, eine Anordnung zu konstruieren, bei der es keine zwei Dominosteine gibt, die ein 2x2 Quadrat bilden.
	O.b.d.A. sei der erste Stein links oben in der Ecke mit der längeren Seite zur oberen Kante des Schachbrettes ausgerichtet (siehe Abbildung \ref{anfang1}).
	\begin{figure}
		\begin{subfigure}{0.33\textwidth}
			\begin{tikzpicture}[scale=0.35]
		\draw (0,0) grid (8,8);
		\node (A) at (-0.5,0.5) {H};
		\node (B) at (-0.5,1.5) {G};
		\node (C) at (-0.5,2.5) {F};
		\node (D) at (-0.5,3.5) {E};
		\node (E) at (-0.5,4.5) {D};
		\node (F) at (-0.5,5.5) {C};
		\node (G) at (-0.5,6.5) {B};
		\node (H) at (-0.5,7.5) {A};
		\node (1) at (0.5,-0.5) {1};
		\node (2) at (1.5,-0.5) {2};
		\node (3) at (2.5,-0.5) {3};
		\node (4) at (3.5,-0.5) {4};
		\node (5) at (4.5,-0.5) {5};
		\node (6) at (5.5,-0.5) {6};
		\node (7) at (6.5,-0.5) {7};
		\node (8) at (7.5,-0.5) {8};
		\filldraw[thick,draw=blue,fill=gray] (0,8) rectangle (2,7);
\end{tikzpicture}
			\subcaption{Anfangsanordnung, andere A1 bedeckende Anordnungen sind symmetrisch}
			\label{anfang1}
		\end{subfigure}
		\begin{subfigure}{0.33\textwidth}
			\begin{tikzpicture}[scale=0.35]
	\draw (0,0) grid (8,8);
	\node (A) at (-0.5,0.5) {H};
	\node (B) at (-0.5,1.5) {G};
	\node (C) at (-0.5,2.5) {F};
	\node (D) at (-0.5,3.5) {E};
	\node (E) at (-0.5,4.5) {D};
	\node (F) at (-0.5,5.5) {C};
	\node (G) at (-0.5,6.5) {B};
	\node (H) at (-0.5,7.5) {A};
	\node (1) at (0.5,-0.5) {1};
	\node (2) at (1.5,-0.5) {2};
	\node (3) at (2.5,-0.5) {3};
	\node (4) at (3.5,-0.5) {4};
	\node (5) at (4.5,-0.5) {5};
	\node (6) at (5.5,-0.5) {6};
	\node (7) at (6.5,-0.5) {7};
	\node (8) at (7.5,-0.5) {8};
	\filldraw[thick,draw=blue,fill=gray] (0,8) rectangle (2,7);
	\filldraw[thick,draw=blue,fill=gray] (0,7) rectangle (1,5);
\end{tikzpicture}
			\subcaption{Einzige Möglichkeit, einen Stein auf B2 hinzulegen}
			\label{b21}
		\end{subfigure}
		\begin{subfigure}{0.32\textwidth}
			\begin{tikzpicture}[scale=0.35]
	\draw (0,0) grid (8,8);
	\node (A) at (-0.5,0.5) {H};
	\node (B) at (-0.5,1.5) {G};
	\node (C) at (-0.5,2.5) {F};
	\node (D) at (-0.5,3.5) {E};
	\node (E) at (-0.5,4.5) {D};
	\node (F) at (-0.5,5.5) {C};
	\node (G) at (-0.5,6.5) {B};
	\node (H) at (-0.5,7.5) {A};
	\node (1) at (0.5,-0.5) {1};
	\node (2) at (1.5,-0.5) {2};
	\node (3) at (2.5,-0.5) {3};
	\node (4) at (3.5,-0.5) {4};
	\node (5) at (4.5,-0.5) {5};
	\node (6) at (5.5,-0.5) {6};
	\node (7) at (6.5,-0.5) {7};
	\node (8) at (7.5,-0.5) {8};
	\filldraw[thick, draw=blue,fill=gray] (0,8) rectangle (2,7);
	\filldraw[thick, draw=blue,fill=gray] (0,7) rectangle (1,5);
	\filldraw[thick, draw=blue,fill=gray] (1,7) rectangle (3,6);
	\filldraw[thick, draw=blue,fill=gray] (1,6) rectangle (2,4);
	\filldraw[thick, draw=blue,fill=gray] (2,6) rectangle (4,5);
	\filldraw[thick, draw=blue,fill=gray] (2,5) rectangle (3,3);
	\filldraw[thick, draw=blue,fill=gray] (3,5) rectangle (5,4);
	\filldraw[thick, draw=blue,fill=gray] (3,4) rectangle (4,2);
	\filldraw[thick, draw=blue,fill=gray] (4,4) rectangle (6,3);
	\filldraw[thick, draw=blue,fill=gray] (4,3) rectangle (5,1);
	\filldraw[thick, draw=blue,fill=gray] (5,3) rectangle (7,2);
	\filldraw[thick, draw=blue,fill=gray] (5,2) rectangle (6,0);
	\filldraw[thick, draw=blue,fill=gray] (6,2) rectangle (8,1);
\end{tikzpicture}
			\subcaption{Hier muss es ein 2x2-Quadrat mit zwei Dominosteinen entstehen}
			\label{ende1}
		\end{subfigure}
	\end{figure}
	Wir betrachten nun Feld B2. Der Dominostein muss an dieser Stelle vertikal ausgerichtet sein, da ansonsten ein Quadrat aus zwei Dominosteinen entsteht (siehe Abbildung \ref{b21}).\\
	Der Feld B2 bedeckende Dominostein muss horizontal ausgerichtet sein, da ansonsten ein Quadrat aus zwei Dominosteinen entsteht. Wird nach diesem Schema weiterverfahren, so erhält man Diagramm \ref{ende1}.
	An dieser Stelle muss nun ein Quadrat aus zwei Dominosteinen entstehen, es gibt also keine Anordnung der Dominosteine ohne ein solches Quadrat.
	\newpage
	\section{Aufgabe}
	Die Dezimaldarstellung $SCHLAF$ wird nun umgeformt zu $S*10^5+C*10^4+H*10^3+L*10^2+A*10+F$. Analog wird $FLACHS$ zu $F*10^5+L*10^4+A*10^3+C*10^2+H*10+S$ umgeformt.
	O.b.d.A. sei $SCHLAF>FLACHS$. Dann gilt für die Differenz der beiden Zahlen 
	\begin{align*}
		&SCHLAF-FLACHS\\
		=&S*10^5+C*10^4+H*10^3+L*10^2+A*10+F\\&-(F*10^5+L*10^4+A*10^3+C*10^2+H*10+S)\\
		=&(S-F)*10^5-(S-F)*1+(C-L)*10^4-(C-L)*10^2\\&+(H-A)*10^3-(H-A)*10\\
		=&(S-F)*99999+(C-L)*(10^4-10^2)+(H-A)*(10^3-10)\\
		=&(S-F)*99999+(10*(C-L)+(H-A))*11*10*9\\
		=&(S-F)*271*369+\underbrace{(10*(C-L)+(H-A))}_{\text{zweistellig}}*11*5*3^2*2
	\end{align*}
	Der Teil vor dem Plus lässt sich stets durch 271 teilen, wie aus den Umformungen eindeutig ersichtlich wird.
	Da 271 eine Primzahl ist, lässt sich der zweite Teil nur dann durch 271 teilen, wenn 271 Teil der Primfaktorzerlegung ist oder der zweite Teil 0 wird. Da alle Faktoren kleiner als 271 sind, muss der gesamte zweite Teil 0 werden, damit sich die Summe aus Teil 1 und Teil 2 durch 271 teilen lässt.
	$SCHLAF-FLACHS$ ist also genau dann durch 271 teilbar, wenn der zweite Teil 0 wird, also $C=L$ und $H=A$.
	\newpage
	\section{Aufgabe}
	Ein beliebiger Punkt $P$ habe die Koordinaten $P(x_P|y_P)$. Eine Gerade durch zwei Punkte $P$ und $Q$ heiße $g_{PQ}: y = m_{PQ}*x+c_{PQ}$. Die Länge einer Strecke von $P$ nach $Q$ sei $|PQ|$.
	Wir legen O.b.d.A. ein Koordinatensystem fest, sodass die Punkte an folgenden Stellen liegen (siehe Abbildung \ref{skizze3}):
	\[A(0|0), B(2|0), C(2|2), D(0|2)\]
	Zudem heiße der Winkel $DAE$ $\alpha$.
	Die Gerade $g_{AE}$ hat die Geradengleichung $y = m_{AE}*x$ mit $m_{AE}=tan(\alpha)$. Für $g_{AF}$ gilt analog: $y = tan(\alpha+45^{\circ})*x$.
	Damit erhält man die Punkte $E(2|2tan(\alpha))$ und $F(\dfrac{2}{tan(\alpha+45^{\circ})}|2)$.
	Für die Punkte $H$ und $G$ gilt, dass sie auf einem Kreis mit Mittelpunkt $(1|1)$ und Radius $\sqrt{2}$ liegen. Für sie gilt also:
	\begin{figure}
		\begin{tikzpicture}[scale=2]
		\pgfmathsetmacro{\w}{25}
		\pgfmathsetmacro{\r}{sqrt(2)}
		\coordinate (A) at (0,0);
		\coordinate (B) at (2,0);
		\coordinate (C) at (2,2);
		\coordinate (D) at (0,2);
		\coordinate (eprojektion) at (\w:3);
		\coordinate (fprojektion) at (\w+45:3);
		
		\draw[->] (A) -- (0,3);
		\draw[->] (A) -- node[pos = .16,above] {$\alpha$} (3,0);
		\draw (0.7,0) arc (0:\w:0.7);
		\draw[name path=g1] (A) -- (eprojektion);
		\draw[name path=g2] (A) -- (fprojektion);
		\draw (A) node[below left] {A} rectangle (C) node [above right] {C};
		\path[name path=upwards] (B) node[below right] {B}--(C)--(D) node[above left] {D};
		\draw[name path=circle] (1,1) circle (\r);
		\path[name intersections={of=g1 and upwards, by=E}];
		\path[name intersections={of=g2 and upwards, by=F}];
		\path[name intersections={of=g1 and circle, by=G}];
		\path[name intersections={of=g2 and circle, by=H}];
		
		\draw (E)node[below right]{E}--(F) node[above left] {F};
		\draw (G)node[below right]{G}--(H) node[above left] {H}; 
		\end{tikzpicture}
		\caption{Skizze gemäß Aufgabentext}
		\label{skizze3}
	\end{figure}
	\begin{align*}
		(x-1)^2+(y-1)^2&=2\\
		x^2-2x+1+y^2-2y+1&=2\\
		x^2-2x+y^2-2y&=0
	\end{align*}
	Nun berechnen wir die Schnittpunkte von $AG=AE$ und dem Kreis.
	\begin{align*}
		y&=tan(\alpha)*x\\
		0&=x^2-2x+(tan(\alpha)*x)^2-2*tan(\alpha)*x\\
		0&=x*(x-2+tan(\alpha)^2*x-2tan(\alpha))\\
		\intertext{x=0 ist eine bereits bekannte Lösung}
		0&=x-2+tan(\alpha)^2*x-2tan(\alpha)\\
		(tan(\alpha)^2+1)*x&=2+2tan(\alpha)\\
		x&=\dfrac{2+2tan(\alpha)}{tan(\alpha)^2+1}\\
		x&=\dfrac{2+2\frac{sin(\alpha)}{cos(\alpha)}}{1+\frac{sin(\alpha)^2}{cos(\alpha)^2}}&&\left|*\frac{cos(\alpha)^2}{cos(\alpha)^2}\right.\\
		x&=\dfrac{2cos(\alpha)^2+2sin(\alpha)*cos(\alpha)}{sin(\alpha)^2+cos(\alpha)^2}\\
		x&=2cos(\alpha)*(sin(\alpha)+cos(\alpha))
	\end{align*}
	\begin{align*}
		\intertext{Analog gilt für den Schnittpunkt von $AH=AF$ und dem Kreis}
		x&=2cos(\alpha+45^{\circ})*(sin(\alpha+45^{\circ})+cos(\alpha+45^{\circ}))\\
		\intertext{Mit den Additionstheoremen erhalten wir}
		x&=2*(cos(\alpha)*cos(45^{\circ})-sin(\alpha)*sin(45^{\circ}))\\
		&*(cos(\alpha)*cos(45^{\circ})-sin(\alpha)*sin(45^{\circ})+sin(\alpha)*sin(45^{\circ})+cos(\alpha)*cos(45^{\circ}))\\
		x&=2*(cos(\alpha)*\frac{\sqrt{2}}{2}-sin(\alpha)*\frac{\sqrt{2}}{2})*(2*cos(\alpha)*\frac{\sqrt{2}}{2})\\
		x&=2cos(\alpha)*(cos(\alpha)-sin(\alpha))
	\end{align*}
	Für die $y$-Koordinate von $G$ gilt: $y = tan(\alpha)*x = \frac{\sin(\alpha)}{cos(\alpha)}$.
	Damit erhalten wir Punkt \[G\left(2cos(\alpha)*(sin(\alpha)+cos(\alpha))\left|2sin(\alpha)*(sin(\alpha)+cos(\alpha))\right.\right)\]
	Für die $y$-Koordinate von $H$ gilt:
	\begin{align*}
	y &= tan(\alpha+45^{\circ})*x = \frac{\sin(\alpha+45^{\circ})}{cos(\alpha+45^{\circ})}*x\\
	&= \frac{sin(\alpha)*sin(45^{\circ})+cos(\alpha)*cos(45^{\circ})}{cos(\alpha)*cos(45^{\circ})-sin(\alpha)*sin(45^{\circ})}*x\\ 
	&= \frac{sin(\alpha)+cos(\alpha)}{cos(\alpha)-sin(\alpha)}*x\\ 
	&= \frac{sin(\alpha)+cos(\alpha)}{cos(\alpha)-sin(\alpha)}*2cos(\alpha)*(cos(\alpha)-sin(\alpha)) \\
	&= 2cos(\alpha)*(cos(\alpha)+sin(\alpha))
	\end{align*}
	Damit erhalten wir außerdem Punkt \[H\left(2cos(\alpha)*(cos(\alpha)-sin(\alpha))\left|2cos(\alpha)*(cos(\alpha)+sin(\alpha))\right.\right)\]
	Nun berechnen wir die Steigungen der Geraden $EF$ und $GH$.
	Zu zeigen: 
	\[m_{EF}=\frac{y_E-y_F}{x_E-x_F} = m_{GH}=\frac{y_G-y_H}{x_G-x_H}\]
	Zunächst berechnen wir $m_{EF}$ mit $E(2|2tan(\alpha)=2\frac{sin(\alpha)}{cos(\alpha)})$ und $F(\frac{2}{tan(\alpha+45^{\circ})}|2)$ und mit der oben erhaltenen Identität $tan(\alpha+45^{\circ})=\frac{sin(\alpha)+cos(\alpha)}{cos(\alpha)-sin(\alpha)}$ ergibt sich der Punkt zu $F(2\frac{cos(\alpha)-sin(\alpha)}{sin(\alpha)+cos(\alpha)}|2)$
	\begin{align*}
		m_{EF}&=\frac{y_E-y_F}{x_E-x_F}=\frac{2\frac{sin(\alpha)}{cos(\alpha)}-2}{2-2\frac{cos(\alpha)-sin(\alpha)}{sin(\alpha)+cos(\alpha)}}&&\left|\frac{0.5cos(\alpha)*(sin(\alpha+cos(\alpha))}{0.5cos(\alpha)*(sin(\alpha)+cos(\alpha))}\right.\\
		&=\frac{(sin(\alpha)+cos(\alpha))*(sin(\alpha)-cos(\alpha))}{cos(\alpha))*(sin(\alpha)+cos(\alpha))-cos(\alpha)(cos(\alpha)-sin(\alpha)}\\
		&=\frac{sin(\alpha)^2-cos(\alpha)^2}{2*cos(\alpha)*sin(\alpha)}
	\end{align*}
	Nun berechnen wir $m_{GH}$.
	\begin{align*}
		m_{GH}&=\frac{y_G-y_H}{x_G-x_H}\\
		&=\frac{2sin(\alpha)*(sin(\alpha)+cos(\alpha))-2cos(\alpha)*(cos(\alpha)+sin(\alpha))}{2cos(\alpha)*(sin(\alpha)+cos(\alpha))-2cos(\alpha)*(cos(\alpha)-sin(\alpha))}\\
		&=\frac{(sin(\alpha)-cos(\alpha))*(sin(\alpha)+cos(\alpha))}{cos(\alpha)*(sin(\alpha)+cos(\alpha)-cos(\alpha)+sin(\alpha))}\\
		&=\frac{sin(\alpha)^2-cos(\alpha)^2}{2*cos(\alpha)*sin(\alpha)}
	\end{align*}
	Daraus folgt: \[m_{EF}=\frac{sin(\alpha)^2-cos(\alpha)^2}{2*cos(\alpha)*sin(\alpha)}=\frac{sin(\alpha)^2-cos(\alpha)^2}{2*cos(\alpha)*sin(\alpha)}=m_{GH}\]
	Die beiden Geraden $g_{EF}$ und $g_{GH}$ sind also parallel.
\newpage
\section{Aufgabe}
	Wir multiplizieren $\sqrt{2}$ mit $10^n$ ($n\in \mathbb{N}$) und runden ab. Damit erhalten wir eine natürliche Zahl, bestehend aus einer 1 und den ersten $n$ Nachkommastellen von $\sqrt{2}$. Diese Zahl nennen wir $s_n$. Für die ersten $n$ Nachkommastellen von $\sqrt{2}$ gilt, dass es keine bessere untere Annäherung an $\sqrt{2}$ gibt. Erhöht man also die letzte Stelle um 1, so ist die Zahl schon größer als $\sqrt{2}$. Wir nutzen diese Bedingung und schreiben
\begin{align}
    s_n<\sqrt{2}*10^{n}<(s_n+1)\\
	s_n^2<2*10^{2n}<(s_n+1)^2
\end{align}
	In einer solchen Ungleichung (für $k$ Nachkommastellen mit $k\in \mathbb{N}$) sei $s_k^2$ unsere Annäherung $A_k$ für $k$ Stellen, $2*10^{2k}$ unser Grenzwert $G_k$ für $k$ Stellen und $(s_k+1)^2$ sei $B_k$.
	Es gilt stets: $A_k<G_k<B_k$ mit $A_k,G_k,B_k\in\mathbb{N}$.
	Nun nehmen wir an, dass nach diesen ersten $n$ Stellen von $\sqrt{2}$\ \ $n+1$ Nullen kommen. Auch für diese angenommenen $2n+1$ Nachkommastellen von $\sqrt{2}$ müsste obige Bedingung gelten. Wir multiplizieren unsere Annäherung von $\sqrt{2}$ diesmal mit $10^{2n+1}$, nennen sie $s_{2n+1}$ und quadrieren erneut:
	\[s_{2n+1}^2<2*10^{4n+2}<(s_{2n+1}+1)^2\]
	Nun ist aber $s_{2n+1}$ aufgrund der $n+1$ angehängten Nullen einfach das $10^{n+1}$-fache von $s_n$.
	\[(10^{n+1}*s_n)^2<2*10^{4n+2}<(10^{n+1}*s_n+1)^2\]
	Nun betrachten wir die Differenz $G_{2n+1}-A_{2n+1}$.
	\begin{align*}
		G_{2n+1}-A_{2n+1}&<B_{2n+1}-A_{2n+1}\\
		2*10^{4n+2}-(10^{n+1}*s_n)^2&<(10^{n+1}*s_n+1)^2-(10^{n+1}*s_n)^2\\
		2*10^{4n+2}-10^{2n+2}*s_n^2&<10^{2n+2}*s_n^2+2*10^{n+1}*s_n+1-10^{2n+2}*s_n^2\\
		10^{2n+2}*\left(2*10^{2n}-s_n^2\right)&<2*10^{n+1}*s_n+1\\
		2*10^{2n}-s_n^2&<\frac{2*10^{n+1}*s_n+1}{10^{2n+2}}\\
		G_n-A_n&<\frac{2*s_n}{10^{n+1}}+\frac{1}{10^{2n+2}}\\
		\intertext{Mit Gleichung (4.1) folgt}
		G_n-A_n&<\frac{2*\sqrt{2}*10^{n}}{10^{n+1}}+\frac{1}{10^{2n+2}}\\
		G_n-A_n&<\frac{2*\sqrt{2}}{10}+\frac{1}{10^{2n+2}}\\
		G_n-A_n&<1
	\end{align*}
	Da sowohl $G_n$ als auch $A_n$ natürliche Zahlen sind und ihre Differenz kleiner als 1 ist, müssen die beiden identisch sein.
	\begin{align*}
	G_n&=A_n\\
	2*10^{2n}&=s_n^2\\
	2*10^{2n}&=(\lfloor\sqrt{2}*10^n\rfloor)^2\\
	\sqrt{2}*10^n&=\lfloor\sqrt{2}*10^n\rfloor
	\end{align*}
	Das würde heißen, dass $\sqrt{2}*10^n$ eine ganze Zahl ist. Das trifft aber offensichtlich nicht zu, sodass unsere Annahme ad absurdum geführt wurde und jede Folge von $n+1$ aufeinanderfolgenden Nullen nach den ersten $n$ Ziffern in der Dezimaldarstellung von $\sqrt{2}$ nicht die bestmögliche Annäherung an $\sqrt{2}$ sein kann und daher in der Dezimaldarstellung von $\sqrt{2}$ nicht vorkommt.\\
	q.e.d.\\
	s.d.g.
\end{document}